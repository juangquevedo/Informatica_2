\documentclass[12pt]{article}
\usepackage[spanish]{babel}
\usepackage{amsmath}
\usepackage{graphicx}
\usepackage{url}
\usepackage{fancyhdr}
\pagestyle{fancy}
\usepackage{hyperref}

\begin{document}

\begin{center}
\bf{\sc\Huge Proyecto de investigación interrupciones}\\
\end{center}
\vspace{60pt}
\begin{center}
\bf{\sc\Huge universidad de Antioquia}\\
\end{center}
\vspace{120pt}
\begin{center}
\bf{\sc\Huge Juan Guillermo Quevedo Cifuentes }\\
\end{center}
\vspace{120pt}
\begin{center}
\bf{\sc\Huge Facultad de ingeniería}
\end{center}
\vspace{5pt}
\begin{center}
\bf{\sc\Huge Informatica II}
\end{center}
\vspace{5pt}
\begin{center}
\bf{\sc\Huge 2020}\\
\end{center}
\newpage

\large
Cuando se utiliza hardware se puede tener la necesidad de vigilar algún periférico para ejecutar alguna instrucción, pero ¿cómo lo podemos lograr esto? Ya que, si se implementa dentro del código que se ejecuta se tendría que parar el código cíclicamente para revisar la señal del periférico, esto le cuesta tiempo al programa que podría utilizar en ejecutar las otras instrucciones. Hay una solución para este problema que son las interrupciones, lo que hacen es reservar una dirección de memoria para que cuando el procesador reciba una señal determinada, pause el programa que está ejecutando y realice las instrucciones que se encuentran en esa dirección de memoria, cuando termine de ejecutar estas instrucciones reanudara las instrucciones interrumpidas por la señal del periférico. Las interrupciones no solo se utilizan en el hardware también se utilizan en el software, esto ya que en ocasiones se necesita recolectar información del disco duro, entonces para no parar el programa a la espera de esta información se utiliza una interrupción para indicar que la información ya está disponible.\cite{Sistemasintegrados}

\vspace{10}

\large
Antes de las interrupciones se utilizaba el sondeo, es la manera simple para resolver el problema, que es que el microprocesador le pregunte constantemente al periférico que, si necesita su atención, pero este método era ineficiente ya que gastaba tiempo y capacidad del microprocesador que se podrían estar siendo usadas por las instrucciones, además que entre más periféricos controle el microprocesador mas tiempo va a mal gastar el procesador en preguntarle si necesita ser usado. Al ver este problema se idearon las interrupciones que permiten que no sea el procesador el que le pregunte al periférico si necesita su atención, si no, que el periférico sea el que le pide al microprocesador que le preste atención.\cite{interruptsmadeeasy}

\newpage

\bibliographystyle{plain} 

\bibliography{Bibliografia.bib}

\end{document}

